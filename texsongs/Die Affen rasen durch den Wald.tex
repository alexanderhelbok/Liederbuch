% !TeX root = Liederbuchtest.tex
\begin{finalbox}{\artist}
\titlee{Die Affen rasen durch den Wald}
\end{finalbox}

\begin{enumerate}
\myverse{ Die \acc{C} Affen \acc{Am} rasen \acc{C} durch den \acc{Am} Wald, \\ 
der \acc{C} eine \acc{Am} macht den \acc{C} andern \acc{Am} kalt. }

\myrefrain[]{Refrain:}{ Die ganze \acc{G7} Affenbande \acc{C} brüllt: \\ 
{\normalfont\bfseries|:} \acc{C7} Wo ist die \acc{F} Kokosnuss, \\ 
wo ist die \acc{Am} Kokosnuss, \\ 
wer hat die \acc{G7} Kokosnuss ge\acc{C}klaut ? {\normalfont\bfseries:|} }

\myverse{ Die \acc{C} Affen\acc{Am}mama \acc{C} sitzt am \acc{Am} Fluss \\ 
und \acc{C} angelt \acc{Am} nach der \acc{C} Kokos\acc{Am}nuss. \Reff }

\myverse{ Dem \acc{C} Affen\acc{Am}papa \acc{C} macht’s Ver\acc{Am}druss, \\ 
er \acc{C} hätt’ so \acc{Am} gern die \acc{C} Kokos\acc{Am}nuss. \Reff }

\myverse{ Der \acc{C} Affen\acc{Am}onkel, \acc{C} welch ein \acc{Am}Graus, \\ 
reißt \acc{C} ganze \acc{Am} Urwald\acc{C}bäume \acc{Am} aus. \Reff }

\myverse{ Die \acc{C} Affen\acc{Am}tante \acc{C} kommt von \acc{Am} fern, \\ 
sie \acc{C} ißt die \acc{Am} Kokos\acc{C} nuss so \acc{Am} gern. \Reff }

\myverse{ Der \acc{C} Affen\acc{Am}milc\acc{hm}ann, \acc{C} dieser \acc{Am} Knilch, \\ 
der \acc{C} wartet \acc{Am} auf die \acc{C} Koko\acc{Am}smilch. \Reff }

\myverse{ Das \acc{C} Affen\acc{Am}baby \acc{C} voll Ge\acc{Am}nuss \\ 
hält \acc{C} in der \acc{Am} Hand die \acc{C} Kokos\acc{Am}nuss. \\ 
Die ganze \acc{G7} Affenbande \acc{C} brüllt: \\ 
|: \acc{G7} Da ist die \acc{F} Kokosnuss, \\ 
da ist die \acc{Am} Kokosnuss, \\ 
es hat die \acc{G7} Kokosnuss ge Cklaut!:| }

\myverse{ Die \acc{C} Affen\acc{Am}mama \acc{C} schreit: \acc{Am} Hurra, \\ 
die \acc{C} Kokos\acc{Am} nuss ist \acc{C} wieder \acc{Am} da! \Reff }

\myverse{ Und \acc{C} die Mo-\acc{Am}-ral von \acc{C} der Ge-\acc{Am}-schicht: \\ 
Klaut \acc{C} keine \acc{Am} Kokos-\acc{C}-nüsse nicht, \\ 
weil sonst die \acc{G7} ganze Bande \acc{C} brüllt: \\ 
\acc{C7} Wo ist die \acc{F} Kokosnuss, ... }

\end{enumerate}