\title{Franziskus} 

\begin{enumerate}
\myverse{ \acc{C} Ich ziehe froh und zufrieden durch die Lande, \\ 
ob Regen, Schnee oder Sonnen-\acc{G}-schein. \\ 
Ich bleibe stets guter Dinge \\ 
und ein Liedlein ich singe. \\ 
Warum \acc{G7} sollte es auch etwa anders \acc{C} sein? \\ 
Ja seht ich habe einen Vater dort im Himmel; \\ 
er sorgt für mich, denn ich bin sein \acc{G} Kind. \\ 
Was soll ich Sorgen mir machen, \\ 
lieber ist mir das Lachen, \\ 
und so \acc{G7} pfeif’ ich alle Sorgen in den \acc{C} Wind. }

\myverse{ \acc{C} Ich ziehe froh und zufrieden durch die Lande \\ 
und frage niemals nach Geld und \acc{G} Gut. \\ 
Und wenn mir fehlt Trank und Speise, \\ 
ein Gewand für die Reise, \\ 
Gott gibt alles und dazu noch frohen \acc{C} Mut. \\ 
Er hat gesagt: Seht die Vögel und die Blumen, \\ 
sie ernten nie, doch ich sorg’ für \acc{G} sie. \\ 
Und auch für euch will ich sorgen, \\ 
denkt nicht ängstlich an morgen, \\ 
und so \acc{G7} pfeif’ ich munter meine Melo-\acc{C}-die. }

Pfeifen \acc{C}-\acc{G}-\acc{C}-\acc{G}-\acc{G7}-\acc{C}

\myverse{ \acc{C} Ich ziehe froh und zufrieden durch die Lande, \\ 
vor lauter Glück mir das Herze \acc{G} lacht. \\ 
Die Berge dort und die Wälder, \\ 
Täler, Wiesen und Felder, \\ 
alles hat der Herr zur Freude mir er-\acc{C}-dacht. \\ 
Die Menschen sind alle meine lieben Brüder, \\ 
es gebe Gott ihnen frohen \acc{G} Sinn. \\ 
Die ganze Welt soll ihn loben, \\ 
unsern Vater dort oben, \\ 
und so \acc{G7} pfeif’ ich immer fröhlich \\ 
vor mich \acc{C} hin. }

\end{enumerate}